\documentclass{article}

\usepackage[utf8]{inputenc}
\usepackage[left = 2cm, top = 2cm, right = 3cm, bottom=3cm]{geometry}
\usepackage[ampersand]{easylist}

\begin{document}
\begin{easylist}[itemize]
& Quem s�o os usu�rios?
&& Clientes de lojas e supermercardos.
& Por que eles utilizariam?
&& Para proporcionar uma experi�ncia diferente ao comprar produtos em loja f�sica. 
& Quais necessidades eles t�m?
&& Organizar melhor suas compras e gastos.
& Qual o perfil dos usu�rios?
&& Jovens que constumam aderir tecnologias novas.
& H� diferentes pap�is para os usu�rios?
&& N�o, eles s�o os clientes da loja, ou, dependendo das funcionalidades, consumidores que est�o 
pesquisando produtos em diversas lojas.
& Quais tarefas eles precisam ou gostariam de fazer?
&& Criar uma lista de compras
&& Ver os pre�os dos produtos
&& Listar itens para o carrinho (por c�digo ou por marca��o na lista de compras)
& Quais novidades voc� imagina que eles querem?
&& Adicionar valor previsto (Escondido da tela ao fazer as compras).
&& Consultar hist�rico de compras (por listas cadastradas ou NF-e).
&& Ver porcentagem de valor gasto de acordo com o esperado.
&& Ver encarte de forma online.
&& Ser avisado de promo��es.
&& Adicionar produtos por c�digo visual.
& Quais outras oportunidades do uso?
&& Visualizar card�pio de restaurantes ao entrar no estabelecimento e, adicionando a op��o de enviar lista de compras, fazer pedidos pelo cardapio (necess�rio dados de usu�rio).
& Qual o contexto organizacional e social de uso?
&& Lojas - sejam em shoppings ou ruas - e supermercados grandes.
& Quando, onde e como eles v�o utilizar?
&& Inicialmente, apenas dentro de lojas.
& Com quem eles v�o interagir e cooperar?
&& Apenas com a lista de produtos.
& O que motiva o uso?
&& Busca por experi�ncias novas em compras
& Quais s�o os processos de neg�cio da organiza��o?
&& A organiza��o precisa integrar o banco de dados com o aplicativo ou cadastrar os produtos.
Fazendo com que os clientes possam ver o que � oferecido, prosibilitando atrair novos clientes ao negocio.
\newpage
& Procurar sistemas com prop�sitos semelhantes
&& Relativo a pontos essenciais da ideia 
&&& O Google Keep fornece um servi�o de cadastrar lista de compras em forma de anota��o
&& Relativo a pontos adicionais da ideia
&&& As Lojas Americanas tem um aplicativo onde � poss�vel ler o pre�o dos produtos por c�digo de barra.
& Anotar pontos positivos e outros destaques
& Pe�a a um colega para observar o seu desenho e apontar pontos positivos e negativos
& Surgiram novas ideias?
\end{easylist}
\end{document}