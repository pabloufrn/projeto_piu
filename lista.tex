\documentclass{article}

\usepackage[utf8]{inputenc}
\usepackage[left = 2cm, top = 2cm, right = 3cm, bottom=3cm]{geometry}
\usepackage[ampersand]{easylist}

\begin{document}
\begin{easylist}[itemize]
& Quem são os usuários?
&& Pessoas interessadas em sair para lugares fora do habitual
& Por que eles utilizariam?
&& Diferente de outras aplicações serão mostradas locais de eventos diários, e para os estabelecimentos permanentes será possível a visualização de produtos em estoque. 
& Quais necessidades eles têm?
&& Obter informações filtradas pelo gosto de cada usuário sobre lugares e eventos na cidade.
& Qual o perfil dos usuários?
&& Todos aqueles interessados em aproveitar eventos culturais e outros lugares na cidade
& Há diferentes papéis para os usuários?
&& Eles podem visualizar ou cadastrar, permanentemente ou não, eventos e lugares.
& Quais tarefas eles precisam ou gostariam de fazer?
&& Ter de maneira facilitada informações sobre lugares e eventos ṕróximos.
& Quais novidades você imagina que eles querem?
&& Além de conhecer um evento ou estabelecimento ser possível saber quanto será possível gastar nele.
& Quais outras oportunidades do uso?
&& É possível que os usuários criem grupos para gerenciar eventos públicos ou privados. 
& Qual o contexto organizacional e social de uso?
&& Quando, onde e como eles vão utilizar?
&&& É possível usar a qualquer momento desde que estejam com internet.
&& Com quem eles vão interagir e cooperar?
&&& Com quem criaria os eventos e cooperar com os grupos de afinidade.
&& O que motiva o uso?
&&& Praticidade em obter informações sobre eventos e produtos.
&& Quais são os processos de negócio da organização?
&&& Facilidade na organização de tarefas
& Procurar sistemas com propósitos semelhantes
&& Existe o google maps, porém ele não indica eventos apenas estabelecimentos sem a facilidade de se ver os preços. Também tem o Sympla que promove a venda de ingressos para eventos porém não mostra eventos próximos ao usuário.
& Anotar pontos positivos e outros destaques
&&  Existe o google maps, porém ele não indica eventos apenas estabelecimentos sem a facilidade de se ver os preços. 
&& Também tem o Sympla que promove a venda de ingressos para eventos porém não mostra eventos próximos ao usuário.
% ----------
% IDEIA 2 |
% ----------
& Quem são os usuários?
&& Clientes de lojas e supermercardos.
& Por que eles utilizariam?
&& Para proporcionar uma experiência diferente ao comprar produtos em loja física. 
& Quais necessidades eles têm?
&& Organizar melhor suas compras e gastos.
& Qual o perfil dos usuários?
&& Jovens que constumam aderir tecnologias novas.
& Há diferentes papéis para os usuários?
&& Não, eles são os clientes da loja, ou, dependendo das funcionalidades, consumidores que estão 
pesquisando produtos em diversas lojas.
& Quais tarefas eles precisam ou gostariam de fazer?
&& Criar uma lista de compras
&& Ver os preços dos produtos
&& Listar itens para o carrinho (por código ou por marcação na lista de compras)
& Quais novidades você imagina que eles querem?
&& Adicionar valor previsto (Escondido da tela ao fazer as compras).
&& Consultar histórico de compras (por listas cadastradas ou NF-e).
&& Ver porcentagem de valor gasto de acordo com o esperado.
&& Ver encarte de forma online.
&& Ser avisado de promoções.
&& Adicionar produtos por código visual.
& Quais outras oportunidades do uso?
&& Visualizar cardápio de restaurantes ao entrar no estabelecimento e, adicionando a opção de enviar lista de compras, fazer pedidos pelo cardapio (necessário dados de usuário).
& Qual o contexto organizacional e social de uso?
&& Lojas - sejam em shoppings ou ruas - e supermercados grandes.
& Quando, onde e como eles vão utilizar?
&& Inicialmente, apenas dentro de lojas.
& Com quem eles vão interagir e cooperar?
&& Apenas com a lista de produtos.
& O que motiva o uso?
&& Busca por experiências novas em compras
& Quais são os processos de negócio da organização?
&& A organização precisa integrar o banco de dados com o aplicativo ou cadastrar os produtos.
Fazendo com que os clientes possam ver o que é oferecido, prosibilitando atrair novos clientes ao negocio.
\newpage
& Procurar sistemas com propósitos semelhantes
&& Relativo a pontos essenciais da ideia 
&&& O Google Keep fornece um serviço de cadastrar lista de compras em forma de anotação
&& Relativo a pontos adicionais da ideia
&&& As Lojas Americanas tem um aplicativo onde é possível ler o preço dos produtos por código de barra.
& Anotar pontos positivos e outros destaques
&& Controle finaceiro.
&& Melhor experiencia em compras (Não esquecer mais os items que tinha que comprar)
\end{easylist}
\end{document}